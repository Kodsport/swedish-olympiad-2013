\problemname{Räkneord}

Monika skriver ner alla räkneord mellan 1 och $N$. Sedan sorterar hon orden alfabetiskt och konkatenerar dem till en enda lång sträng. Skriv ett program som hittar bokstav nummer $K$, $K+1$ och $K+2$ i Monikas sträng.

För att undvika problem med teckenkodning byter vi ut å och ä mot a. Vidare använder vi formerna etthundra, ettusen (observera stavningen), enmiljon och enmiljard, men bara när siffran 1 står {\em ensam} före mängdordet (ej inräknat tidigare mängdord). När siffran 1 däremot står sist i ett förled som betecknar {\em flera} tusen, miljoner eller miljarder så skriver vi för enkelhets skull alltid ``en'': tjugoentusen, etthundraenmiljoner, tvahundraattioenmiljarder etc. Allra sist i ett räkneord skriver vi dock alltid siffran 1 som ``ett''.

En tabell med de grundläggande räkneorden och några exempel på längre ord bringar förhoppningsvis klarhet i reglerna:

\begin{tabular}{|r|l||r|l|} \hline
{\bf Tal} & {\bf Räkneord} & {\bf Tal} & {\bf Räkneord} \\ \hline
1 & ett & 10 & tio\\
2 & tva & 20 & tjugo\\
3 & tre & 30 & trettio\\
4 & fyra & 40 & fyrtio\\
5 & fem & 50 & femtio\\
6 & sex & 60 & sextio\\
7 & sju & 70 & sjuttio\\
8 & atta & 80 & attio\\
9 & nio & 90 & nittio\\
11 & elva & 100 & etthundra\\
12 & tolv & 198 & etthundranittioatta\\
13 & tretton & 201 & tvahundraett\\
14 & fjorton & 1121 & ettusenetthundratjugoett\\
15 & femton & $581\,743$ & femhundraattioentusensjuhundrafyrtiotre \\
16 & sexton & $51\,101\,001$ & femtioenmiljoneretthundraentusenett\\
17 & sjutton & $162\,500\,020$ & etthundrasextiotvamiljonerfemhundratusentjugo\\
18 & arton & $1\,002\,001\,004$ & enmiljardtvamiljonerettusenfyra\\
19 & nitton & $91\,011\,091\,000$ & nittioenmiljarderelvamiljonernittioentusen\\ \hline
\end{tabular}

Och som en ytterligare kontrollmöjlighet ger vi längden på Monikas sträng för några olika $N$:

\begin{tabular}{|r|r|}\hline
{\bf N} & {\bf Strängens längd} \\ \hline
999 & 16260 \\
9999 & 235600 \\
99999 & 2908000 \\
999999 & 37425000 \\
9999999   & 472250000 \\
99999999 & 5319500000\\
999999999 & 61585000000\\
9999999999 & 722850000000 \\
99999999999 &	7834500000000 \\
999999999999 &86744000000000 \\ \hline
\end{tabular}


\section*{Indata}
En rad med två heltal $N$ och $K$, där $1 \leq N < 10^12$ och $1 \leq K \leq L-2$, där $L$ är längden på den bildade strängen.

\section*{Utdata}
En rad med tre bokstäver utan åtskiljande blanksteg: bokstav nummer $K$, $K+1$ och $K+2$ i strängen. 

\section*{Poängsättning}
Din lösning kommer att testas på en mängd testfallsgrupper.
För att få poäng för en grupp så måste du klara alla testfall i gruppen.

\noindent
\begin{tabular}{| l | l | p{12cm} |}
  \hline
  \textbf{Grupp} & \textbf{Poäng} & \textbf{Gränser} \\ \hline
  $1$   & $15$       & $N \leq 2000 $\\ \hline
  $2$   & $20$       & $N \leq 5*10^5 $\\ \hline
  $3$   & $30$       & $N = 999\,999\,999\,999 $\\ \hline
  $4$   & $35$       & Inga ytterligare begränsningar. \\ \hline
\end{tabular}

\section*{Förklaring till första exemplet}

Monikas sträng, skriven med 40 bokstäver per rad, är

\begin{verbatim}
artonattaelvaettfemfemtonfjortonfyranion
ittonsexsextonsjusjuttontiotjugotjugoett
tjugofemtjugofyratjugosextjugotretjugotv
atolvtretrettontva
\end{verbatim}
