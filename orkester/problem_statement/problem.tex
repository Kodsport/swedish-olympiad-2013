\problemname{Orkesteroptimering}

Problemlösar-orkestern är en samling programmerarmusiker i full gång med att planera sin kommande turné. Den viktigaste uppgiften är att maximera oväsendet de kan åstadkomma i varje låt, detta för att uppnå optimal publikuppskattning.

Orkestern består av $N$ musiker och har bett dig att optimera en låt som består av $M$ takter. Varje musiker har endast övat på de takter han/hon funnit intressanta och är inte kapabel att spela resten av takterna. Varje gång som en musiker spelar en takt så görs detta med oväsendet $V=1/(X+1)$ oväsen-enheter, där $X$ är antalet takter som musikern spelat tidigare i stycket. En musiker spelar alltså svagare och svagare ju fler takter han/hon spelar, vilket självklart beror på den ansträngning det tar att lyfta sitt instrument.

Den totala mängden oväsen för låten definieras som summan av mängden oväsen i alla takter. Mängden oväsen i en takt beror i sin tur endast på den musiker som spelar starkast, alltså det maximala $V$ för de som spelar i takten eller $0$ om ingen gör det. Din uppgift är att räkna ut hur mycket oväsen som kan åstadkommas, givet att orkestern spelar optimalt.

\emph{Notera att alla musiker spelar lika bra, det enda som har betydelse för mängden oväsen en musiker kan åstadkomma är antalet takter han/hon har spelat.}

\section*{Indata}
Den första raden består av två heltal, $N$ och $M$. $N$ rader följer, var och en beskrivande en musiker. Dessa rader börjar med ett heltal $T_i$, antal takter som musiker i övat in, och följs av $T_i$ stycken heltal som beskriver vilka takter musikern övat på. Varje takt representeras med ett tal mellan $1$ och $M$.

\section*{Utdata}
Skriva ut en rad med ett flyttal, den maximala mängd oväsen som kan åstadkommas i låten. Ett absolut fel mindre än $10^{-5}$ betraktas som korrekt.

\section*{Delpoäng}
\begin{itemize}
\item För 20\% av poängen gäller att $1 \leq N \leq 20$ och $1 \leq M \leq 20$.
\item För 80\% av poängen gäller att $1 \leq N \leq 1000$ och $20 < M \leq 400$ (TODO: Kalibrera indatastorlekar).
\end{itemize}

\section*{Exempel}
TODO: Förklara exempelindata och förtydliga körtidsgränser (summan av $T_i$).
